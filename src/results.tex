
Research Findings:
This section has been split into nine sections to simplify the comparison of the different materials, and the respective manufacturing workflows associated with them, as well as the properties of the materials themselves. The nine sections are as follows:

\begin{itemize}[itemsep=2mm]
    \item Tensile Strength
    \item Elasticity
    \item Surface Quality
    \item Thermal Properties
    \item Precision
    \item Cost
    \item Time Consumption
    \item Complexity in Manufacturing Process
    \item Environmental Impact
\end{itemize}


\section{Tensile Strength}

    A key attribute that determines a material's capacity to resist pulling forces without breaking is its
    tensile strength. A number of differences were found when 3D-printed PLA (polylactic acid) and 6061-T6 aluminum were compared, emphasizing the distinctive mechanical properties of each material.

    PLA is a bioplastic composed of renewable resources like corn starch and from a technical data sheet
    made by BCN3D Technologies, an ultimate tensile strength of 70 MPa was reported from them. %(link) 
    The value shown is the highest possible stress PLA can withstand before breaking down to tension. Due to
    PLA has a relatively lower tensile strength, being a polymer and having a layer-by-layer deposition while 3D printing, can result in the formation of weak areas along the contact points. Despite its lower
    tensile strength compared to metals, PLA can be used in situations where mechanical strength is not as important and instead can be used when specified components have to be made. Examples of what
    components were made through 3D printing are shown below.
    
    %[insert images of Chassis Body Corners, Suspension Rocker & Boogie, Chassis-Rocker Connector]

    On the other hand, according to a material data sheet published by ASM International %(link)
    , 6061-T6 aluminum, a commonly used alloy in several industrial applications, showed a significantly greater
    ultimate tensile strength of 310 MPa. The composition of the alloy and the T6 tempering process, which
    includes heat treatment and artificial aging to improve the mechanical properties of the alloy, is responsible for its high tensile strength. For structural components and applications that need high
    amounts of strength, 6061-T6 aluminum is the ideal material due to its better tensile strength. A good
    example can be the motor-wheel shaft which transmits torque from the motor to the wheel hub and from
    the wheel hub to the wheel. In addition, there is the motor cover which supports the weight of the rover.
    This can be shown below.

    %[insert images of Shaft and Motor Cover working (can be an image from Fusion360)]

    The wide range in tensile strength between PLA and 6061-T6 aluminum shows the need for choosing a material according to the particular requirements of a given application. PLA has advantages in terms of
    manufacturing and environmental impact (mentioned in Section ….), but its mechanical limits restrict its
    application to less demanding structural tasks. Whereas, when mechanical strength is crucial, 6061-T6
    aluminum's durability and reliability make it essential. This comparison emphasizes how important it is
    for engineering and industrial projects to consider both mechanical qualities and application requirements when choosing materials.
