The purpose of this study aimed to compare the possibility of employing 3D printed PLA components for the rover chassis and other essential parts with conventionally manufactured 6061-T6 aluminium components. Finding an economical and dependable manufacturing technique became important for student teams with usually limited resources.

Tensile strength, elasticity, surface quality, thermal properties, precision, cost, time consumption, manufacturing process complexity, and environmental impact were the main comparison factors in the study. To find the best manufacturing process for rover components, the study carefully examined the factors above.

The results presented helpful information on real-world uses and constraints of PLA 3D printing, helping future project determine the most appropriate manufacturing strategy that corresponded with particular needs and limitations. The purpose of this study was to improve and clarify the decision-making process for creating dependable and efficient rover components by analysing the advantages and disadvantages of each manufacturing technique.