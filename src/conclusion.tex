This study examined the viability of manufacturing small-scale rover chassis parts from 3D-printed PLA as
opposed to 6061-T6 aluminum. The analysis showed that while PLA has important benefits in terms of
cost-effectiveness and speedy prototyping, its use is restricted to low-stress applications because of its
mechanical properties, which include a lower tensile strength and a higher degree of elasticity. Whereas,
6061-T6 aluminum offers more strength, precision and thermal stability, making it appropriate for
demanding applications subjected to severe stress.

PLA is suitable for prototyping and non-structural components due to its faster production time and
reduced production costs. However, two major disadvantages are that it is more prone to deformation in
higher temperatures and has inferior durability. In contrast, structural components made of aluminum are
worth the investment due to their long-term resilience and endurance, even with their greater initial cost
and longer production time.

To broaden the comparison, additional 3D printing processes and aluminium alloys should be investigated
in future studies. Further research is also necessary to fully understand the environmental impact and
long-term performance of these materials.

As a result, individual application requirements should be taken into consideration when deciding between
PLA and 6061-T6 aluminum; balancing aspects such as physical properties, cost, production efficiency,
and environmental impact. This study illustrates how important it is to choose the right materials for
engineering projects in order to maximize their performance and sustainability.