This study examined the viability of manufacturing small-scale rover chassis parts from 3D-printed PLA as
opposed to 6061-T6 aluminum. The analysis showed that while PLA has important benefits in terms of
cost-effectiveness and speedy prototyping, its use is restricted to low-stress applications because of its
mechanical properties, which include a lower tensile strength and a higher degree of elasticity. Whereas,
6061-T6 aluminum offers more strength, precision and thermal stability, making it appropriate for
demanding applications subjected to severe stress.

PLA is suitable for prototyping and non-structural components due to its faster production time and
reduced production costs. However, two major disadvantages are that it is more prone to deformation in
higher temperatures and has inferior durability. In contrast, structural components made of aluminum are way more reliable, but oftentimes inaccessible to small organizations due to the high costs of production, and students due to the complexity of the manufacturing process.

Based on the results of this study, and the concrete case of the Ravensburg Weingarten University of Applied Sciences Rover to Mars Project, the 3D printing workflow comes out as a viable option for the development of small-scale mobile robots. The use of PLA for parts such as joints, small housings, and other semi-structural components can significantly reduce the production time and costs of a project, and make the construction of custom designs more accessible to students, and other organizations with limited resources.