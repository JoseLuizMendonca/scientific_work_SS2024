\documentclass[type=report,foot=true,colorhead=true]{rwuthesis} % twoside, BCOR=1cm,
\usepackage{babel}
% https://tex.stackexchange.com/questions/282678/why-does-inputenc-abandon-so-quickly-under-utf8-based-engines
%\usepackage[utf8]{inputenc} % not to use with xelatex

% ============= Dokuinfo =============

\usepackage[pdfusetitle]{hyperref}
\hypersetup{ %https://en.wikibooks.org/wiki/LaTeX/Hyperlinks#Customization
    colorlinks = true, 
    breaklinks, %line breaking in a long hyperlink
    urlcolor=black, %  \url colour
    linkcolor =black, %  \ref colour
    citecolor=rwuvioletlight% \cite colour
}

% ============= Packages =============
\usepackage{color, url}
\usepackage{float} %big H
\usepackage{pdfpages}
\usepackage{wrapfig}
\usepackage{tabularx, multirow, booktabs}
\usepackage{parcolumns}
\usepackage{listings}

%%graphics and figures defined image path
\usepackage{graphicx}
\graphicspath{ {images/} }
\usepackage{caption} %https://en.wikibooks.org/wiki/LaTeX/Floats,_Figures_and_Captions
\usepackage{subcaption} % multiple figures with subcaptions

% Bibliography
\usepackage{csquotes}
\usepackage{comment}
\usepackage[ 
% https://www.overleaf.com/learn/latex/Bibliography_management_with_biblatex#Reference_guide
% https://www.overleaf.com/learn/latex/Natbib_citation_styles
% https://www.bibtex.com/e/entry-types/
    backend=biber, % biber backend
    natbib=true, % customising citations
    sorting=nty, % sort name, title, year
    style=numeric % https://de.overleaf.com/learn/latex/Biblatex_bibliography_styles
]{biblatex}

% math packages
% \usepackage{amsfonts}
% \usepackage{amsmath}
% \usepackage{amssymb}
\usepackage{MnSymbol} % Math Symbol Font

% ============= additional settings or packages =============
\usepackage{lipsum} % random text generator with \lipsum

%% date format in text
\usepackage[nodayofweek]{datetime}
\newdateformat{mydate}{\twodigit{\THEDAY}{ }\shortmonthname[\THEMONTH] \THEYEAR}

%% line pitch
\usepackage[onehalfspacing]{setspace} %singlespacing. onehalf-, double-

% Increase number of captions available
%https://tex.stackexchange.com/questions/186981/is-there-a-subsubsubsection-command

% Caption of figures/tables continuous
\usepackage{chngcntr}
\counterwithout{figure}{chapter}
\counterwithout{table}{chapter}

% The tocdepth counter decides to which depth down the entries appear in the ToC
% https://tex.stackexchange.com/questions/291307/how-to-hide-show-section-levels-in-the-table-of-contents
% \setcounter{secnumdepth}{4}
% \setcounter{tocdepth}{4}
% \lstset{numberbychapter=false}

% Document information
\newcommand{\batitle}{PLA Additive Manufacturing Chassis Parts for Small Scale Rovers}

\title{\batitle}
\author{Jose Luiz S. Mendonça}
\authormail{js-223959@rwu.de}
\matnumber{35756}
\secondauthor{Zion Smuts}
\secondauthormail{zion.smuts@rwu.de} 
\secondmatnumber{15640111}
\fordegree{Bachelor of Engineering}
%\firstreviewer{Prof. Dr. rer. nat. Markus Pfeil}
%\firstreviewermail{markus.pfeil@rwu.de}
\supervisor{Prof. Dr. rer. nat. Markus Pfeil}
\supervisormail{markus.pfeil@rwu.de}
%\secondreviewer{Dipl.-Ing., Dr.techn. Wilfried Wessner}
%\secondreviewermail{wilfried.wessner@linutronix.de}
\degreecourse{E-Mobility and Green energies}
\faculty{Electrical Engineering}
\date{\today}

% ==========================
\begin{document}
\maketitle

\newpage
\tableofcontents
\newpage
%\onehalfspacing

\chapter{Abstract}
% Context of this paper: in the context of the Canadian international rover challenge,Teams will simulate what it would be like as an early colony on an extraterrestrial planet. Teams will bring their prototype rover to help them accomplish challenging scenarios. The rovers will be faced with completing various tasks that future rovers will be expected to perform. These include traversing varying terrain, autonomous operations, operating a dexterous arm and much more! 

% the question this paper discusses is, is 3d printing a viable option for the production of rover parts, such as joints, axles, and other typical chassis parts?

% The question will be analyzed from our perspective as a team participating in the challenge, and also from a more general research perspective.

% Write the abstract!

	In the context of the Canadian International Rover Challenge, teams made up of students from universities from diverse countries must build a prototype rover to simulate what it would be like as an early colony on an extraterrestrial planet. The rovers will be faced with completing various tasks that future rovers will be expected to perform. These include traversing varying terrain, autonomous operations, operating a dexterous arm and much more!

	The chassis of each rover is a critical component that must be robust and reliable and must be of easy manufacture, as the teams have limited resources and time to build their rovers. Due to the prototypical nature of the rovers, a lot of custom parts are needed. Traditional manufacturing methods such as milling and turning can be used to produce these parts, but are time-consuming and require expensive machinery, which is not always available to student teams.

	3D PLA printing parts with "hobby grade" 3d printers is a great way to prototype small batches of parts, but the produced parts are sometimes considered fragile, or precise enough due to the nature of the production process. However, modern-day 3D printers have come a long way, and the quality of the parts produced by these machines has improved significantly. Research is needed to determine if such a workflow is a viable option for the production of rover parts for such a project, or if indeed, more traditional manufacturing methods are still better suited for such an application.

\chapter{Introduction}
    % writing something
    
\chapter{Results}
\chapter{Discussion}
\chapter{Conclusion}

%% !TeX program = LuaLaTeX
\section*{whatever}
\subsection*{eu não sei o que estou fazendo}

% test

%\sloppy % set sloppy tolerances
% \printbibliography[heading=bibintoc]
%\printbibliography[keyword=pic, title={Pictures only}]
\end{document}
