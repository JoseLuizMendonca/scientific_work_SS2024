The findings of this study, 6061-T6 aluminum had a tensile strength of roughly 310 MPa, but PLA
components had a tensile strength of just 70 MPa. This distinction highlighted the limitation of PLA in
high-stress applications where mechanical strength was important. PLA was found to have higher
elasticity, as displayed by its Young’s modulus of 3.2 GPa, which is lower than that of aluminum at 68.9
GPa. This suggests that PLA was more flexible and could have been used in low-stress scenarios.

Also, what was found in previous research was that the results were consistent. This indicated that PLA
was not as mechanically superior to 6061-T6 aluminum. For instance, we discussed in [Results] that
aluminum alloys have greater tensile strength and endurance in structure applications. This comparison
emphasizes how important it is to choose materials according to the particular requirements of the
application.

The outcomes imply that although PLA 3D printing has many benefits in terms of price, speed, and
versatility, it is not an ideal choice for parts that need to have a high degree of mechanical strength. PLA
can only be used for less demanding structural purposes due to its higher flexibility and lower tensile
strength. On the other hand, the robustness of 6061-T6 aluminum makes it perfect for important parts in
the rover design, assuring performance and dependability under demanding circumstances.
One limitation of this study was its focus on a single type of 3D printer and particular methods for treating
aluminum. To provide a more thorough comparison, future research should take into account a wider
range of 3D printing technologies and alternative aluminum alloys. Additionally, there has not been much
research done on how environmental elements like humidity and temperature can impact PLA’s
characteristics.

In order to understand more about PLA components’ long-term durability in many environmental
scenarios, more research is required. Studies may also look into the viability of hybrid manufacturing
techniques, which combine the advantages of traditional manufacturing techniques with 3D printing for
quick prototyping and high-strength component production. Furthermore, studies into more
environmentally friendly and biodegradable 3D printing materials may provide long-term substitutes for
PLA and aluminum.

In conclusion, the specific requirements of the task at hand will determine whether to use CNC-machined
6061-T6 aluminium or 3D printed PLA. For non-industrial and prototype components, 3D printing offers
flexibility, cost savings, and swift development capabilities. However, 6061-T6 aluminum is still necessary
for applications needing greater strength, precision, and lifespan. This study showcases how important it is
to consider the effects of manufacturing efficiency, mechanical qualities, and the environment when it
comes to choosing materials for engineering projects